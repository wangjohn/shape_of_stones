%!TEX root = paper.tex

The Discrete Fourier transform is defined as

\begin{align*}
  \boldsymbol{\hat{x}}_k& \coloneqq \mathcal{F}_k(\boldsymbol{x})\\
  & \coloneqq \sum_{n=0}^{N-1} e^{-2\pi i k (n/N)} \boldsymbol{x}_n \qquad \text{for} \; k = 0, \dotsc, N-1
\end{align*}

And the inverse transform is defined as

\begin{align*}
  \boldsymbol{x}& \coloneqq \mathcal{F}_n^{-1}(\boldsymbol{\hat{x}})\\
  & \coloneqq \frac{1}{N} \sum_{k=0}^{N-1} e^{2\pi i k (n/N)} \boldsymbol{\hat{x}}_k \qquad \text{for} \quad n = 0, \dotsc, N-1
\end{align*}

The FFT allows us to calculate derivatives easily with the following algorithm (a detailed derivation of the algorithm can be found in \cite{trefethen_2000})

\begin{enumerate}
  \item $\bhat{x} = \mathcal{F}(\boldsymbol{x})$
  \item $\bhat{w}_k = ik \bhat{x}_k$ , for $k = 0, \dotsc, N-1$
  \item Because of a lack of symmetry in the Fouerier frequencies, if p is odd, then we set $\bhat{w}_{N/2} = 0$.
  \item Finally
    \[
      \frac{\partial^p \boldsymbol{x}}{\partial s^p} = \mathcal{F}^{-1}(\bhat{w})
    \]
\end{enumerate}