Stones in riverbeds and on the ocean floor have very distinct shapes and contours. These stones tend to be smooth and flat, but the reason for this is not obvious. This paper examines how erosion affects the shape of stones. In particular, we examine two different mechanisms for stone erosion, a discrete one and a continuous one, and produce models for each one.

The first mechanism we study is the chipping of stones. One can imagine stones in a riverbed being tossed around and colliding with other stones. We create a discrete, polygonal model of a stone and a means of chipping the stone through a shear force. This model predicts that stones will tend to be flat and elongated. Our simple model of the chipping process can result in relatively good estimations of stone shape.

The second mechanism we study is the wearing down of stones by gradual erosion. This mechanism appears when stones are gradually worn away by sand, water, or some other agent. We start of with a simple smooth model of the stone using spectral methods with the Fast Fourier Transform. This leads to problems, since the size of the stone decreases over time and changes shape, so we need to resample. Most of our work involves designing a method for this resampling. We are then able to demonstrate our continuous model on a few simple problems, namely the processes of smoothing a stone and the process of erroding a stone from the bottom.
