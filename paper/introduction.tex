Stones in riverbeds and on the ocean floor have very distinct shapes and contours. These stones tend to be smooth and flat, but the reason for this is not obvious. This paper examines how erosion affects the shape of stones. In particular, we examine two different mechanisms for stone erosion and produce a model for each mechanism.

The first mechanism we study is the chipping of stones. One can imagine stones in a riverbed being tossed around and colliding with other stones. We create a discrete, polygonal model of a stone and a means of chipping the stone through a shear force. This model predicts that stones will tend to be flat and elongated.

The second mechanism we study is the wearing down of stones by gradual erosion.
